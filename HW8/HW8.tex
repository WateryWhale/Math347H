\documentclass[12pt, a4paper]{article}
\usepackage[a4paper, total={6.5in, 8in}]{geometry}
\usepackage{amsmath,amsthm,amssymb}
\usepackage{MnSymbol}
\usepackage{wasysym}
\usepackage{lipsum}% http://ctan.org/pkg/lipsum
\renewcommand{\qedsymbol}{$\blacksquare$}
\title{Math 347 HW 8}
\author{Dylan Chiu}
\begin{document}
    \maketitle
    \subsection*{Exercise 3.6.13}
        \textit{If $(a_k)_{k\in\mathbb{N}}$ is a Cauchy seqence in $\mathbb{R}$, show that, for any $\epsilon > 0$,
        there exists a subsequence $(a_{k_j})_{j\in\mathbb{N}}$ so that $|a_{k_j} - a_{k_j+1} | < \frac{\epsilon}{2^{j+1}}$
        for $j\in\mathbb{N}$}
        \begin{proof}
            fix $\epsilon > 0$

            Since $(a_k)$ is Cauchy, $\exists N_1\in \mathbb{N}, \forall n,m \ge N_1, |a_n-a_m| < \frac{\epsilon}{2^{1+1}}$

            Pick $a_{k_1} = a_{N_1}$. Similarly, $\exists N_2\in \mathbb{N}, \forall m \ge N_2, |a_{k_1}-a_m| < \frac{\epsilon}{2^{2+1}}$

            Pick $a_{k_2} = a_{N_2}$. Again, $\exists N_3\in \mathbb{N}, \forall m \ge N_3, |a_{k_2}-a_m| < \frac{\epsilon}{2^{3+1}}$ and pick $a_{k_3} = a_{N_3}$
            
            Continuing inductively we select $a_{k_j} = N_j$

        \end{proof}
    \subsection*{Exercise 3.6.25}
        \textit{Show that a subset of $\mathbb{R}$ is closed iff it contains all of its accumulation points.}
        \begin{proof}
            $\rightarrow$ Suppose $S\subseteq\mathbb{R}$ and $S$ is closed.
            
            Assume $S$ does not contain all of its accumulation points to derive a contradiction.

            Since $S$ is closed $S^c$ is open.

            This means $\forall x \in S^c, \exists \epsilon>0 s.t B_\epsilon(x) \subseteq S^c$

            Let x be an accumulation point of $S$ that is in $S^c$ 
            
            Then $\forall \epsilon > 0, B_\epsilon(x)\backslash \{x\}\cap S \not = \emptyset$

            Let $y \in B_\epsilon(x)\backslash \{x\}\cap S$

            But since $S^c$ is open $y \in S^c$, a contradiction.

            $\leftarrow$ Suppose $S\subseteq\mathbb{R}$ and $S$ contains all of its accumulation points.
            
            Assume $S$ is not closed to derive a contradiction

            Then $S^c$ is not open which means $\exists x \in S^c \text{ such that } \forall \epsilon>0$,  $B_\epsilon(x) \not\subseteq S^c$

            This is equivalent to $\exists x \in S^c, \forall \epsilon>0$, $ B_\epsilon(x) \cap S = B_\epsilon(x)\backslash\{x\} \cap S \not = \emptyset$ (note $x\not\in S$ )

            But then $x$ is an accumulation point of $S$ that is not in $S$, a contradiction.
     \end{proof}
     \pagebreak
     \subsection*{Exercise 3.6.26}
        Let $C$ be the Cantor set.
        \begin{enumerate}
            \item \textit{$C$ set is closed}
                \begin{proof}
                    $C^c$ is a union of open intervals between $[0,1]$ which implies $C^c$ is open. (Exercise 3.6.24)

                    Therefore $C$ is closed
                \end{proof}
            \item \textit{$C$ is uncountable}
                \begin{proof}
                    We proceed with a typical diagonalization argument.

                    Suppose that $C$ is countable. Then we can create a list of the elements of $C$.\\ 
                    $C$ consists of all numbers in [0,1] whose ternary expansions have only 0's and\\ (possibly infinite) 2's. (part 3)

                    $x_1 = 0.d_1^1d_2^1d_3^1d_4^1\mathellipsis$\\
                    $x_2 = 0.d_1^2d_2^2d_3^2d_4^2\mathellipsis$\\
                    $x_3 = 0.d_1^3d_2^3d_3^3d_4^3\mathellipsis$\\
                    $\vdots$

                    Construct $x=d_1d_2\mathellipsis$ not in the list by setting $d_n= 0$ if $d_n^n = 2$ and $d_n = 2$ if $d_n^n = 0$\\
                    But this is contradicts our assumption that we can list the elements of $C$. \\
                    Therefore $C$ is not countable
                \end{proof}
            \item \textit{$C$ consists of all numbers in the closed interval [0, 1]
            whose ternary expansion consists of only 0's and 2's and may end
            in infinitely many 2's}
                \begin{proof}
                    Consider the ternary representation of $x\in (1/3,2/3)$

                    Write $1/3$ as $0.1=0.0222\mathellipsis$ and  $2/3$  as $0.2$

                    Then $x=0.1d_2d_3\mathellipsis$
                    
                    Removing this middle third means we have now removed all $x$ where $x$ has a $d_1 = 1$
                    
                    Meaning $x\in C_1 = [0,1]\backslash (1,3) \implies x=0.0d_2d_3d_4\mathellipsis$ or $x=0.2d_2d_3d_4\mathellipsis$

                    Now consider $x \in (1/9, 2/9)$. $x=0.01d_3d_4\mathellipsis$

                    Simiilarly,  $x \in (5/9, 8/9)$ has the form $x=0.21d_3d_4\mathellipsis$

                    Removing these two open intervals removes all x where x has the form $0.d_11d_3d_4\mathellipsis$

                    Continuing inductively, on the nth step we remove all x where $d_n=1$.


                \end{proof}
            \item \textit{Every point of $C$ is an accumulation point of $C$}
                \begin{proof}
                    Consider $x\in C$ and $(x-\epsilon, x+\epsilon)\backslash\{x\}$ where $\epsilon > 0$

                    Construct an element of $C$ in $(x-\epsilon, x+\epsilon)\backslash\{x\}$ as follows:

                    Note that $2*3^{-n}$ is a number in ternary with $d_i = 0$ for i from 1 to n-1 and $d_{n}=2$\\
                    For example, $2*3^{-4} = 0.0002$\\
                    Also note that the ternary expansion of $x$ consists only of 0's and 2's. 


                    \textbf{Case 1:} Ternary expansion of $x$ terminates

                    Idea is to append arbitrarily many 0's to $x$ and then a 2\\
                    Let $d_n^\epsilon$ be the first non zero digit of $\epsilon$ and $m$ be the length of the ternary form of $x$.\\
                    $k = max(m,n)$\\
                    $x + 2*3^{-k-1}< x+\epsilon$ and it is an element of $C$ since it contains only 0s and 2s

                    \textbf{Case 2:} $x$ ends in infinite 2's

                    Idea is to turn a 2 from the ``tail" of $x$ into a 0\\
                    Let $d_n^\epsilon$ be the first non zero digit of $\epsilon$\\
                    Let $d_m^x$ be the last 0 of the ternary expansion of $x$.\\
                    $k=max(m, n)$\\
                    $a = 2*3^{k+1}$\\
                    Then $x-a > x-\epsilon$ and it is an element of $C$
                    
                    Every neighborhood of $x$ contains an element in $C$. Therefore $x$ is an accumulation point. 
                    
                \end{proof}
                \item \textit{The set {\normalfont[0, 1]}$\backslash$Cantor set is a dense subset of {\normalfont[0, 1]}.}
                
                    \begin{proof}
                        Let $a,b\in[0,1]$ and $b > a$.\\
                        Let $C_n$ be the nth iteration of the Cantor set construction\\
                        $C_n$ is a union of open interverals of length $\frac{1}{3^n}$\\
                        Choose n large enough that $b-a < \frac{1}{3^n}$\\
                        Then $[a,b]$ is not completely contained in any one interval of $C_n$\\
                        Then there is $x\in[a,b]$ but $x \not\in C_n$\\
                        $\implies x\not\in C$\\
                        $\implies x\in[0,1]\backslash C$\\
                    \end{proof}
        \end{enumerate}
\end{document}