\documentclass[12pt, a4paper]{article}
\usepackage[a4paper, total={6.5in, 8in}]{geometry}
\usepackage{amsmath,amsthm,amssymb}
\usepackage{MnSymbol}
\usepackage{wasysym}
\title{Math 347 HW 7}
\author{Dylan Chiu}
\begin{document}
    \maketitle
    \subsection*{Exercise 3.5.5.}
        \textit{Show that the limit of a convergent sequence is unique.}
        \begin{proof}
            To derive a contradiction suppose $a,b$ are distinct limits of a convergent sequence $(a_k)$
            Then definition of convergence states: 
            Given any arbitrary $r \in \mathbb{Q}, \exists N_1, N_2 \in \mathbb{N} s.t \forall n \ge N_1, |a_n - a| < r$ and $\forall m \ge N_2 |a_m - b| < r$

            Let $N=max(N_1, N_2), r = \frac{|a-b|}{4}$ (notice $a\not = b$ implies $r > 0$)

            Then $\forall n \ge N, |a_n - a| < r$ and $|a_n - b| < r.$

            $|a_n-b| = |b - a_n|$

            $|a_n - a| + |b - a_n| < 2r = \frac{|a-b|}{2}$
            
            $|(a_n-a)+(b-a_n)| \le |a_n - a| + |b - a_n|$ (triangle $\not =$)

            $|b-a| = |a-b| \le |a_n - a| + |b - a_n| < \frac{|a-b|}{2}$

            $|a-b| < \frac{|a-b|}{2}$. A contradiction 
            
        \end{proof}
    \subsection*{Exercise 3.5.10}
        \textit{Show that, with addition and multiplication defined as above, C is a commutative ring with 1.}
        \begin{proof}
            (already shown in class that addition and multiplication are closed)
            Now show that addition is associative, commutative, has identity, and has inverses.
            Show that multiplication is asssociative, commutative, and has identity. 

            Let $(a_k)_{k\in \mathbb{N}} \in C$

            Multiplication and addition are obviously associative and commutative by properties of the rationals. 

            Additive identity = $(0,0,\mathellipsis)$

            Additive inverse $-(a_k) = (-a_1, -a_2, \mathellipsis)$

            Multiplicative identity = $(1,1,\mathellipsis)$
        \end{proof}
        \pagebreak
    \subsection*{Exercise 3.5.13.}
    \textit {Show that if a Cauchy sequence does not converge to 0, all the terms of the sequence eventually have the same sign}
        \begin{proof}
            If a Cauchy sequence $(a_k)_{k \in \mathbb{N}}$ does not converge to 0, then $\exists r > 0, N_1 \in \mathbb{N}, \forall n \ge N_1, |a_n| \ge r$ (Lemma 3.5.12)

            Since $(a_k)$ is Cauchy, $\exists N_2 \in \mathbb{N} s.t |a_n-a_m| < \frac{r}{2}, \forall n,m \ge N$

            Let $ N = max(N_1, N_2)$

            To derive a contradiction, suppose $a_n, a_m$ have opposite parity. 

            Then consider $a_n >0, a_m < 0$ 
            
            $|r-a_m| < \frac{r}{2}$

            But then $a_m > 0$, a contradiction. 

        \end{proof}
    \subsection*{Exercise 3.5.15.}
    \textit{Show that $\sim$ defines an equivalence relation on C.}
        \begin{proof}
            $\sim$ is reflexive: $(a_k-a_k)_{k\in N} = (0,0,0..) \in I$

            $\sim$ is symmetric: 
            
            Suppose $(a_k)\sim (b_k)$, then $(a_k-b_k) \in I.$

            Then $(a_k-b_k)$ converges to 0. 

            Then $\forall r \in \mathbb{Q}, \exists N, \forall n \ge N, |a_n-b_n| < r$

            $|a_n-b_n| = |b_n-a_n| < r$

            $(b_k-a_k)$ converges to 0. $(b_k-a_k)$ is in I. 

            then $(b_k) \sim (a_k)$


            $\sim$ is transitive: 

            Suppose $(a_k) \sim (b_k)$ and $(b_k) \sim (c_k)$

            Then $(c_k) \sim (b_k)$ ($\sim$ is symmetric)

            $(a_k - b_k) , (c_k - b_k)$ converge to 0

            $(a_k - b_k) - (c_k - b_k) = (a_k - c_k)$

            I is closed under addition, therefore $(a_k) \sim (c_k)$

        \end{proof}
    \pagebreak
    \subsection*{Exercise 3.5.16.}
        \textit{Show that addition and multiplication are well-defined on} $\mathbb{R}$
        \begin{proof}
            Suppose $(a_k)\sim (a’_k)$ and $(b_k)\sim (b’_k)$

            Addition:

            $(a_k-a’_k) \in I$ and converges to 0

            $(b_k - b’_k) \in I$ and converges to 0. 

            $(a_k-a’_k) + (b_k - b’_k) = (a_k-a’_k + b_k - b’_k) = (a_k+b_k - (a’_k+b’_k)) \in I$ since $I$ is closed under addition

            $(a_k+b_k ) \sim (a’_k+b’_k)$

            Multiplication: We want $(a_k*b_k) \sim (a’_k * b’_k)$

            I.e want $(a_k*b_k-a’_k * b’_k) \in I$

            $a_k*b_k-a’_k * b’_k = a_k*b_k+(b_k*a’_k - b_k*a’_k) - a’k*b’_k$

            $= (a_k*b_k - b_k*a’_k) + (b_k*a’_k - a’_k*b’_k)$

            $=b_k(a_k-a’_k) + a’_k(b_k-b’_k)$

            |$a_k*b_k-a’_k * b’_k| = | b_k(a_k-a’_k) + a’_k(b_k-b’_k)|$

            $< | b_k|*|(a_k-a’_k)|+ |a’_k|*|(b_k-b’_k)|$

            There exists $S,L$ such that $|b_k| \le S$ and $|a’_k| \le L.$ since all Cauchy sequences are bounded. 

            Let $M = max(S,L).$ 

            $|a_k*b_k-a’_k * b’_k| < M(|(a_k-a’_k)|+ |(b_k-b’_k)|)$ 

            $(a_k-a’_k) and (b_k-b’_k) \in I.$

            $|a_k*b_k-a’_k * b’_k| < c_k$ where $c_k$ converges to 0.

            Since $c_k$ converges to 0, $\forall \epsilon > 0, \exists N s.t \forall n\ge N, |c_n| < \epsilon$

            $|a_n*b_n-a’_n * b’_n| < \epsilon$ 

            Thus $a_k*b_k-a’_k * b’_k$ converges to zero.

            $(a_k*b_k) \sim (a’_k * b’_k)$
        \end{proof}

        \subsection*{Exercise 3.5.20}
            \textit{Show that the order relation on R defined above is well defined and makes R an ordered field.}
            \begin{proof}
                Let $(a_k) \sim (a’_k)$ and $(b_k) \sim (b’_k)$ and suppose $(a_k) < (b_k)$

                This means $\exists N \in \mathbb{N} s.t \forall n\ge N, a_n < b_n$
                
                $a_n  < b_n-(b’_n+b’_n)$

                $a_n  < (b_n-b’_n)+(b’_n)$

                $a_n  < |(b_n-b’_n)+(b’_n)|$

                $a_n < r + |b’_n|$

                $|a'_n-(a'_n-a_n)| < r + |b'_n|$

                $|a'_n| - |a'_n-a_n| < r + |b'_n|$

                $|a'_n| < r + |b'_n|$ 
            \end{proof}
\end{document}